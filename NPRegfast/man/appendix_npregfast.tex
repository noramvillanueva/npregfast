\inputencoding{utf8}
\HeaderA{NPRegfast-package}{Nonparametric estimation by using local linear kernel smoothers}{NPRegfast.Rdash.package}
\aliasA{NPRegfast}{NPRegfast-package}{NPRegfast}
\keyword{package}{NPRegfast-package}
\keyword{models}{NPRegfast-package}
\keyword{nonparametric regression}{NPRegfast-package}
\keyword{local linear kernel smoothers}{NPRegfast-package}
\keyword{Interactions factor by curve}{NPRegfast-package}
%
\begin{Description}\relax
This package provides a method for obtain nonparametric estimates using local linear kernel smoothers.

Particular features of the package are facilities for fast smoothness estimation, and the calculation of their first and second derivative. Users can define the smoothers parameters. Confidences intervals calculation is provided by bootstrap methods. Binning techniques were applied to speed up computation in the estimation and testing processes.
\end{Description}
%
\begin{Details}\relax

\Tabular{ll}{
Package: & NPRegfast\\{}
Type: & Package\\{}
Version: & 1.0\\{}
Date: & 2013-01-10\\{}}

\code{NPRegfast} provides functions for nonparametric regression models \code{\LinkA{frfast}{frfast}}, \code{\LinkA{plot.frfast}{plot.frfast}}. The term \code{frfast} is taken to include any nonparametric regression estimated by local lineal kernel smoothers. A number of other functions such \code{\LinkA{summary.frfast}{summary.frfast}} are also provided, for extracting information from a fitted \code{frfast} Object. 

For a listing of all routines in the NPRegfast package type:
\code{library(help="NPRegfast")}. For an overview of the NPRegfast package see NPRegfast-package.
\end{Details}
%
\begin{Author}\relax
Marta Sestelo, Nora M. Villanueva and Javier Roca-Pardiñas.

Maintainer: Marta Sestelo <sestelo@uvigo.es>
\end{Author}
%
\begin{References}\relax
Efron, B. (1979). Bootstrap methods: another look at the jackknife. Annals of Statistics, 7:1-26.

Efron, E. and Tibshirani, R. J. (1993). An introduction to the Bootstrap. Chapman and Hall, London.

Sestelo, M. and Roca-Pardinas, J. (2011). A new approach to estimation of length-weight relationship of \eqn{Pollicipes}{}  \eqn{pollicipes}{} (Gmelin, 1789) on the Atlantic coast of Galicia (Northwest Spain): some aspects of its biology and management. Journal of Shellfish Research, 30(3):939-948.

Wand, M. P. and Jones, M. C. (1995). Kernel Smoothing. Chapman \& Hall, London.

\end{References}
%
\begin{Examples}
\begin{ExampleCode}
## See examples for frfast
\end{ExampleCode}
\end{Examples}



\inputencoding{utf8}
\HeaderA{allotest}{Bootstrap based test for testing an allometric model}{allotest}
%
\begin{Description}\relax
In order to facilitate the choice of a model appropriate to the data while at the same time endeavouring to minimise the loss of information,  a bootstrap-based procedure, that test whether the data can be modelled by an allometric model, was developed.  Therefore, \code{allotest} tests  the null hypothesis of an allometric model taking into account the logarithm of the original variable (\eqn{X^* = log(X)}{} and \eqn{Y^* =log (Y)}{})

\eqn{H_0 = m(X^*) =  a^*+ b^* X^*}{} 

\eqn{vs.}{} general hypothesis \eqn{H_1}{} being \eqn{m}{} an unknown nonparametric function; or analogously,

\eqn{H_1: m(X^*)= a^*+ b^* X^* + g(X^*)}{}

with \eqn{g(X^*)}{} being another unkown function. To implement this test we have used the wild bootstrap. 
\end{Description}
%
\begin{Usage}
\begin{verbatim}
allotest(formula, data = data, nboot=100)
\end{verbatim}
\end{Usage}
%
\begin{Arguments}
\begin{ldescription}
\item[\code{formula}] an object of class \code{formula}: a sympbolic description of the model to be fitted. 

\item[\code{data}] a data frame or matrix containing the model response variable and covariates required by the \code{formula}.

\item[\code{nboot}] number of bootstrap repeats.

\end{ldescription}
\end{Arguments}
%
\begin{Value}
An object is returned with the following elements:
\begin{ldescription}
\item[\code{value}] the p-value of the test.
\item[\code{statistic}] the value of the test statistic.
\end{ldescription}
\end{Value}
%
\begin{Author}\relax
Marta Sestelo, Nora M. Villanueva and Javier Roca-Pardiñas.
\end{Author}
%
\begin{Examples}
\begin{ExampleCode}
library(NPRegfast)
data(barnacle)
allotest(DW~RC:F,data=barnacle)
\end{ExampleCode}
\end{Examples}

\inputencoding{utf8}
\HeaderA{frfast}{Fitting nonparametric models}{frfast}
%
\begin{Description}\relax
\code{frfast} is used to fit nonparametric models by using local linear kernel smoothers.
\end{Description}
%
\begin{Usage}
\begin{verbatim}
frfast(formula, data = data, model = 'np', h = -1.0, nh = 30, weights = NULL, kernel = 'epanech', p = 3, kbin = 100, nboot = 500, rankl = NULL, ranku = NULL)
\end{verbatim}
\end{Usage}
%
\begin{Arguments}
\begin{ldescription}
\item[\code{formula}] an object of class \code{formula}: a sympbolic description of the model to be fitted. The details of model specification are given under 'Details'.
 
\item[\code{data}]  a data frame or matrix containing the model response variable and covariates required by the \code{formula}. 
  
\item[\code{model}] type model used: \code{model = 'np'}  nonparametric regression model with local linear kernel smoothers, \code{model = 'allo'} the  allometric model.
 
\item[\code{h}] the kernel bandwidth smoothing parameter. Large values of bandwidth make smoother estimates, smaller values of bandwidth make less smooth estimates. The default is a bandwidth compute by cross validation.

\item[\code{nh}] integer number of equally-spaced bandwidth on which the \code{h} is discretised, to speed up computation. 

\item[\code{weights}] prior weights on the data.

\item[\code{kernel}] character which determines the smoothing kernel. By default \code{kernel = 'epanech' }, this is, the Epanechnikov density function. Also, several types of kernel funcitons can be used:  triangular and Gaussian density function, with 'triang' and 'gaussian' term, respectively.

\item[\code{p}] degree of polynomial used.  Its value must be greater than or equal to the value of drv. The default value is of degree is drv + 1.

\item[\code{kbin}] number of binning nodes over which the function is to be estimated. 
 
\item[\code{nboot}] number of bootstrap repeats.

\item[\code{rankl}] number or vector specifying the minimum value for an interval at which to search the \code{x} value which maximizes the estimate, first or second derivative  (for each level). The default is the minimum data value.

\item[\code{ranku}] number or vector specifying the maximum value for an interval at which to search the \code{x} value which maximizes the estimate, first or second derivative  (for each level). The default is the maximum data value.
\end{ldescription}
\end{Arguments}
%
\begin{Details}\relax
The models fit by \code{frfast} function are specified in a compact symbolic form. The \bsl{}\textasciitilde{}operator is basic in the formation of such models. An expression of the form y \textasciitilde{} model is interpreted as a specification that the response y is modelled by a linear predictor specified symbolically by model. The terms themselves consist of variable and factor names separated by : operators. Such a term is interpreted as the interaction of all the variables and factors appearing in the term.
\end{Details}
%
\begin{Value}
An object is returned with the following elements:
\begin{ldescription}
\item[\code{x}] vector of values of the grid points at which model is to be estimate.

\item[\code{p}] matrix of values of the grid points at which to compute the estimate, their first and second derivative.

\item[\code{pl}] lower values of  95\% confidence interval for the estimate, their first and second derivative.

\item[\code{pu}] upper values of  95\% confidence interval for the estimate, their first and second derivative.

\item[\code{diff}] differences between the estimation values of a couple of levels (i. e. level 2 - level 1). The same procedure for their first and second derivative.

\item[\code{diffl}] lower values of 95\% confidence interval for the differences between the estimation values of a couple of levels. It is performed for their first and second derivative.

\item[\code{diffu}] upper values of 95\% confidence interval for the differences between the estimation values of a couple of levels. It is performed for their first and second derivative.

\item[\code{nboot}] number of bootstrap repeats.

\item[\code{n}] total number of data

\item[\code{dp}] degree of polynomial used.

\item[\code{h}] the kernel bandwidth smoothing parameter.

\item[\code{fmod}] factor's level for each data.

\item[\code{xdata}] original x values

\item[\code{ydata}] original y values

\item[\code{w}] weights on the data.

\item[\code{nf}] number of levels.

\item[\code{kbin}] number of binning nodes over which the function is to be estimated.

\item[\code{pvalue}] it is \code{NULL} when the nonparamentric model is fitted. However, if the p-value of the allometric test. 

\item[\code{max}] value of covariate \code{x} which maximizes the  estimate, first or second derivative.

\item[\code{maxu}] upper value of 95\% confidence interval for the value \code{max}.

\item[\code{maxl}] lower value of 95\% confidence interval for the value \code{max}.

\item[\code{diffmax}] differences between the estimation of \code{max} for a couple of levels (i. e. level 2 - level 1). The same procedure for their first and second derivative.

\item[\code{diffmaxu}] upper value of 95\% confidence interval for the value \code{diffmax}.

\item[\code{diffmaxl}] lower value of 95\% confidence interval for the value \code{diffmax}.

\item[\code{statistic}] the value of the test statistic.

\item[\code{repboot}] matrix of values of the grid points at which to compute the estimate, their first and second derivative for each bootstrap repeat.

\item[\code{ranku}] minimum value for an interval at which to search the \code{x} value which maximizes the estimate, first or second derivative  (for each level). The default is the minimum data value.

\item[\code{rankl}] maximum value for an interval at which to search the \code{x} value which maximizes the estimate, first or second derivative  (for each level). The default is the maximum data value.

\item[\code{nmodel}] type model used: \code{model = 1} the nonparametric model, \code{model = 2} the allometric model. 

\item[\code{label}] labels of the variables in the model.

\item[\code{numlabel}] number of labels.

\item[\code{kernel}] character which determines the smoothing kernel.

\item[\code{name}] name of the variables in the model.

\item[\code{formula}] a sympbolic description of the model to be fitted.
\end{ldescription}
\end{Value}
%
\begin{Author}\relax
Marta Sestelo, Nora M. Villanueva and Javier Roca-Pardiñas.
\end{Author}
%
\begin{Examples}
\begin{ExampleCode}
library(NPRegfast)
data(barnacle)


################################################
# Nonparametric regression without interactions
################################################
fit<-frfast(DW~RC,data=barnacle)
fit

summary(fit)

# Change the number of binning nodes and bootstrap replicates
fit<-frfast(DW~RC,data=barnacle,kbin=200,nboot=1000)

##############################################
# Nonparametric regression with interactions
##############################################
fit2<-frfast(DW~RC:F,data=barnacle)
fit2

summary(fit2)
\end{ExampleCode}
\end{Examples}



\inputencoding{utf8}
\HeaderA{globaltest}{Testing the equality of M curves specific to each level}{globaltest}
%
\begin{Description}\relax
\code{globaltest}  can be used to test the equality of the \eqn{M}{} curves specific to each level. This bootstrap based test assumes the  following null hypothesis

\eqn{H_0: m_1 = \ldots = m_M}{}

Note that, if \eqn{H_0}{} is not rejected, then the equality of critical points will also accepted. 

To test the null hypothesis, it is used an statistic, \eqn{T}{}, based on direct nonparametric estimates of the curves. If the null hypothesis is true, the \eqn{T}{} value should be close to zero but is generally greater. The test rule based on \eqn{T}{} consists of rejecting the null hypothesis if \eqn{T > T^{1- \alpha}}{}, where \eqn{T^p}{} is the empirical \eqn{p}{}-percentile of \eqn{T}{} under the null hypothesis. 
\end{Description}
%
\begin{Usage}
\begin{verbatim}
globaltest(formula,data = data, der = NULL, weights = NULL, nboot = 200, h = -1.0, nh = 30, kernel = 'epanech', p = 3, kbin = 100)
\end{verbatim}
\end{Usage}
%
\begin{Arguments}
\begin{ldescription}
\item[\code{formula}] an object of class \code{formula}: a sympbolic description of the model to be fitted.

\item[\code{data}] a data frame or matrix containing the model response variable and covariates required by the \code{formula}.
 
\item[\code{der}] number which determines any inference process. By default \code{der} is \code{NULL}. If this term is \code{0}, the calculate of the differences for maximum point is for the estimate. If it is \code{1} or \code{2}, it is designed for the first or second derivative, respectively.

\item[\code{weights}] prior weights on the data.
 
\item[\code{nboot}] number of bootstrap repeats.

\item[\code{h}] the kernel bandwidth smoothing parameter. Large values of bandwidth make smoother estimates, smaller values of bandwidth make less smooth estimates. The default is a bandwidth compute by cross validation.
 
\item[\code{nh}] integer number of equally-spaced bandwidth on which the \code{h} is discretised, to speed up computation.

\item[\code{kernel}] character which determines the smoothing kernel. By default \code{kernel='epanech'}, this is, the Epanechnikov density function. Also, several types of kernel funcitons can be used:  triangular and Gaussian density function, with \code{'triang'} and \code{'gaussian'} term, respectively.
  
\item[\code{p}] degree of a polynomial.

\item[\code{kbin}] number of binning nodes over which the function is to be estimated.

\end{ldescription}
\end{Arguments}
%
\begin{Value}

The \eqn{T}{} valueand the \eqn{p}{}-value  are returned. Additionally, it is shown the decision, accepted or rejected, of the global test. The null hypothesis is rejected if the \eqn{p}{}-value\eqn{< 0.05}{}.  

\end{Value}
%
\begin{Author}\relax
Marta Sestelo, Nora M. Villanueva and Javier Roca-Pardiñas.
\end{Author}
%
\begin{Examples}
\begin{ExampleCode}

library(NPRegfast)
data(barnacle)

################################################
# Nonparametric regression without interactions
################################################
glocaltest(DW~RC,data=barnacle, der=0)

##############################################
# Nonparametric regression with interactions
##############################################
glocaltest(DW~RC:F,data=barnacle, der=0)

\end{ExampleCode}
\end{Examples}

\inputencoding{utf8}
\HeaderA{localtest}{Testing the equality of critical points}{localtest}
%
\begin{Description}\relax
\code{localtest} can be used to test the equality of the \eqn{M}{} critical points estimated from the respective level-specific curves. Note that, even if the curves and/or their derivatives are different, it is possible for these points to be equal. 
For instance, taking the maxima of the first derivatives into account, interest lies in testing the following null hypothesis

\eqn{H_0: x_{01} = \ldots = x_{0M}}{}
The above hypothesis is true if \eqn{D = x_{0j} - x_{0k} = 0}{} where 

\eqn{(j,k) = arg max_{1\le l < m \le M} |x_{0j} - x_{0k}|}{}

otherwise \eqn{H_0}{} is false. It is important to highlight that, in practice, the true \eqn{x_{0j}}{} are not known, and consequently neither is \eqn{D}{}, so an estimate \eqn{\hat D = \hat x_{0j} − \hat x_{0k}}{} is used, where, in general, \eqn{\hat x_{0l}}{} are the estimates of \eqn{x_{0l}}{} based on the estimated curves \eqn{\hat m_l}{} with \eqn{l = 1, \ldots , M}{}.
Needless to say, since \eqn{\hat D}{} is only an estimate of the true D, the sampling uncertainty of these estimates needs to be acknowledged. Hence, a confidence interval \eqn{(a,b)}{} is created for \eqn{D}{} for a specific level of confidence (e.g., 95\%). 
\end{Description}
%
\begin{Usage}
\begin{verbatim}
localtest(formula, data = data, der = NULL, weights = NULL, nboot = 200, h = -1.0, nh = 30, kernel = 1, p = 3, kbin = 100, ranku = NULL, rankl = NULL)
\end{verbatim}
\end{Usage}
%
\begin{Arguments}
\begin{ldescription}
\item[\code{formula}] an object of class \code{formula}: a sympbolic description of the model to be fitted.
 
\item[\code{data}] a data frame or matrix containing the model response variable and covariates required by the \code{formula}.
 
\item[\code{der}] number which determines any inference process. By default \code{der} is \code{NULL}. If this term is \code{0}, the calculate of the differences for maximum point is for the estimate. If it is \code{1} or \code{2}, it is designed for the first or second derivative, respectively.

\item[\code{weights}] prior weights on the data.

\item[\code{nboot}] number of bootstrap repeats.

\item[\code{h}] the kernel bandwidth smoothing parameter. Large values of bandwidth make smoother estimates, smaller values of bandwidth make less smooth estimates. The default is a bandwidth compute by cross validation.

\item[\code{nh}] integer number of equally-spaced bandwidth on which the \code{h} is discretised, to speed up computation.

\item[\code{kernel}] character which determines the smoothing kernel. By default \code{kernel = 'epanech' }, this is, the Epanechnikov density function. Also, several types of kernel funcitons can be used:  triangular and Gaussian density function, with 'triang' and 'gaussian' term, respectively.
 
\item[\code{p}] degree of a polynomial.

\item[\code{kbin}] number of binning nodes over which the function is to be estimated.

\item[\code{rankl}] number or vector specifying the minimum value for an interval at which to search the \code{x} value which maximizes the estimate, first or second derivative  (for each level). The default is the minimum data value.

\item[\code{ranku}] number or vector specifying the maximum value for an interval at which to search the \code{x} value which maximizes the estimate, first or second derivative  (for each level). The default is the maximum data value.
\end{ldescription}
\end{Arguments}
%
\begin{Value}
The estimate of \eqn{D}{} value is returned and its confidence interval for a specific-level of confidence, i.e. 95\%. Additionally, it is shown the decision, accepted or rejected,  of the local test. Based on the null hypothesis is rejected if a zero value is not within the interval. 
\end{Value}
%
\begin{Author}\relax
Marta Sestelo, Nora M. Villanueva and Javier Roca-Pardiñas.
\end{Author}
%
\begin{Examples}
\begin{ExampleCode}

library(NPRegfast)
data(barnacle)

##############################################
# Nonparametric regression with interactions
##############################################

localtest(DW~RC:F,data=barnacle, der=0)

\end{ExampleCode}
\end{Examples}


\inputencoding{utf8}
\HeaderA{maxp}{Maximum points for the estimate, first and second derivative, with their 95\% confidence intervals}{maxp}
\keyword{\textbackslash{}textasciitilde{}maximum}{maxp}
\keyword{\textbackslash{}textasciitilde{}points}{maxp}
\keyword{\textbackslash{}textasciitilde{}estimation}{maxp}
\keyword{\textbackslash{}textasciitilde{}firstderivative}{maxp}
\keyword{\textbackslash{}textasciitilde{}secondderivative}{maxp}
%
\begin{Description}\relax
Value of covariate \code{x} which maximizes the  estimate, first and second derivative, for each level of the factor. 
\end{Description}
%
\begin{Usage}
\begin{verbatim}
maxp(model, der = NULL)
\end{verbatim}
\end{Usage}
%
\begin{Arguments}
\begin{ldescription}
\item[\code{model}] parametric or nonparametric regression out obtained by \code{\LinkA{frfast}{frfast}} function.

\item[\code{der}] number which determines any inference process. By default \code{der} is \code{NULL}. If this term is \code{0}, the calculate of the maximum point is for the estimate. If it is \code{1} or \code{2}, it is designed for the first or second derivative, respectively.
\end{ldescription}
\end{Arguments}
%
\begin{Value}
An object is returned with the following elements:
\begin{ldescription}
\item[\code{Estimation}] outputs for the estimation where it is included maximum points, and their 95\% confidence intervals (for each level).
\item[\code{First\_der}] outputs for first derivative with maximum points and their 95\% confidence intervals (for each level).
\item[\code{Second\_der}] outputs for second derivative. It means, maximum points and 95\% confidence intervals (for each level).
\end{ldescription}
\end{Value}
%
\begin{Author}\relax
Marta Sestelo, Nora M. Villanueva and Javier Roca-Pardiñas.
\end{Author}
%
\begin{Examples}
\begin{ExampleCode}

library(NPRegfast)
data(barnacle)

################################################
# Nonparametric regression without interactions
################################################
fit<-frfast(DW~RC,data=barnacle)
maxp(fit)
maxp(fit,der=0)
maxp(fit,der=1)
maxp(fit,der=2)
maxp(fit,der=c(0,1))

##############################################
# Nonparametric regression with interactions
##############################################
fit2<-frfast(DW~RC:F,data=barnacle)
maxp(fit2)
maxp(fit2,der=0)
maxp(fit2,der=1)
maxp(fit2,der=2)
maxp(fit2,der=c(0,1))
\end{ExampleCode}
\end{Examples}


\inputencoding{utf8}
\HeaderA{maxp.diff}{Differences between the estimation of maximum points  for two factor's levels}{maxp.diff}
%
\begin{Description}\relax
Differences between the estimation of \code{\LinkA{maxp}{maxp}} for two  factor's levels. \code{maxp}, a returned element  of class \code{\LinkA{frfast}{frfast}}, is the value of covariate \code{x} which maximizes the  estimate, first or second derivative.
\end{Description}
%
\begin{Usage}
\begin{verbatim}
maxp.diff(model, factor2 = NULL, factor1 = NULL, der = NULL)
\end{verbatim}
\end{Usage}
%
\begin{Arguments}
\begin{ldescription}
\item[\code{model}] parametric or nonparametric regression model obtained by \code{\LinkA{frfast}{frfast}} function.

\item[\code{factor1}] first factor's level at which to perform the differences between maximum points.

\item[\code{factor2}] second factor's level at which to perform the differences between maximum points.

\item[\code{der}] number which determines any inference process. By default \code{der} is \code{NULL}. If this term is \code{0}, the calculate of the differences for maximum point is for the estimate. If it is \code{1} or \code{2}, it is designed for the first or second derivative, respectively.

\end{ldescription}
\end{Arguments}
%
\begin{Details}\relax
Differences are calculated by subtracting a factor relative to another (\eqn{factor2 - factor1}{}).  By default \code{factor2} and \code{factor1} are \code{NULL}, so the differences calculated are for all possible combinations between two factors.
\end{Details}
%
\begin{Value}
An object is returned with the following element:
\begin{ldescription}
\item[\code{maxp.diff}] a table with a couple of factor's level where it is used to calculate the differences between maximum points, and their 95\% interval confidence (for the estimation, first and second derivative).
\end{ldescription}
\end{Value}
%
\begin{Author}\relax
Marta Sestelo, Nora M. Villanueva and Javier Roca-Pardiñas.
\end{Author}
%
\begin{Examples}
\begin{ExampleCode}
library(NPRegfast)
data(barnacle)

##############################################
# Nonparametric regression with interactions
##############################################
fit2<-frfast(DW~RC:F,data=barnacle)
maxp.diff(fit2)
maxp.diff(fit2,der=1)
\end{ExampleCode}
\end{Examples}


\inputencoding{utf8}
\HeaderA{plot.diff}{Visualization of the differences between the estimation of curves for two factor's levels}{plot.diff}
%
\begin{Description}\relax
Useful for drawing the differences between the estimation of curves (initial estimate, first or second derivative) for  two factor's levels.  Missing values of factor's levels is not allowed. 
\end{Description}
%
\begin{Usage}
\begin{verbatim}
plot.diff(model, factor2, factor1, der = NULL, est.include = FALSE, xlab = model$name[2], 
ylab = model$name[1], ylim = NULL, main = NULL, col = "black", CIcol = "grey50", ablinecol = "red", abline = TRUE, type = "l", CItype = "l", lwd = 1, CIlwd = 1.5, lty = 1, CIlty = 2, ...)
\end{verbatim}
\end{Usage}
%
\begin{Arguments}
\begin{ldescription}
\item[\code{model}] allometric or nonparametric regression model obtained by \code{\LinkA{frfast}{frfast}} function.

\item[\code{factor2}] 
second factor's level at which to perform the differences between curves.

\item[\code{factor1}] first factor's level at which to perform the differences between curves.

\item[\code{der}] 
number or vector which determines any inference process. By default \code{der} is \code{NULL}. If this term is \code{0}, the calculate of the maximum point is for the estimate. If it is \code{1} or \code{2}, it is designed for the first or second derivative, respectively.

\item[\code{est.include}]  draw the estimates of the model. By default it is \code{FALSE}.

\item[\code{xlab}] a title for the x axis. 

\item[\code{ylab}] 
a title for the y axis.

\item[\code{ylim}] the \code{y} limits of the plot.

\item[\code{main}]  an overall title for the plot.

\item[\code{col}] 
a specification for the default plotting color.

\item[\code{CIcol}] 
a specification for the default confidence intervals plotting color.

\item[\code{ablinecol}] the color to be used for \code{abline}.

\item[\code{abline}]  draw an horizontal line into the plot of the second derivative of the model. By default it is \code{TRUE}.

\item[\code{type}] 
what type of plot should be drawn. Possible types are, \code{p} for points, \code{l} for lines, \code{o} for overplotted, etc. See details in \code{\LinkA{par}{par}}.

\item[\code{CItype}] 
what type of plot should be drawn for confidence intervals. Possible types are, \code{p} for points, \code{l} for lines, \code{o} for overplotted. 

\item[\code{lwd}] the line width, a positive number, defaulting to 1. See details in \code{\LinkA{par}{par}}.

\item[\code{CIlwd}] the line width for confidence intervals, a positive number, defaulting to 1.

\item[\code{lty}] the line type. Line types can either be specified as an integer (0=blank, 1=solid (default), 2=dashed, 3=dotted, 4=dotdash, 5=longdash, 6=twodash). See details in \code{\LinkA{par}{par}}.

\item[\code{CIlty}] the line type for confidence intervals. Line types can either be specified as an integer (0=blank, 1=solid (default), 2=dashed, 3=dotted, 4=dotdash, 5=longdash, 6=twodash). 

\item[\code{...}] 
other options.

\end{ldescription}
\end{Arguments}
%
\begin{Details}\relax
simply produce a plot.
\end{Details}
%
\begin{Author}\relax
Marta Sestelo, Nora M. Villanueva and Javier Roca-Pardiñas.
\end{Author}
%
\begin{Examples}
\begin{ExampleCode}

library(NPRegfast)
data(barnacle)

##############################################
# Nonparametric regression with interactions
##############################################
fit2<-frfast(DW~RC:F,data=barnacle)
plot.diff(fit2,factor2=2,factor1=1)
plot.diff(fit2,factor2=2,factor1=1,der=1,col="red",CIcol="green")
plot.diff(fit2,2,1,der=c(0,1))
\end{ExampleCode}
\end{Examples}


\inputencoding{utf8}
\HeaderA{plot.frfast}{Visualization of frfast objects}{plot.frfast}
%
\begin{Description}\relax
Useful for drawing the estimation, first and second derivative  (for each factor)
\end{Description}
%
\begin{Usage}
\begin{verbatim}
plot(model, fac = NULL, der = NULL, points = TRUE, xlab = model$name[2], ylab = model$name[1], ylim = NULL, main = NULL, col = "black", CIcol = "black", ablinecol = "red", abline = TRUE,type = "l", CItype = "l", lwd = 2, CIlwd = 1,lty = 1, CIlty = 2, ...)
\end{verbatim}
\end{Usage}
%
\begin{Arguments}
\begin{ldescription}
\item[\code{model}] \code{frfast} object.

\item[\code{fac}] number or vector which determines the level to take into account in the plot. By default is \code{NULL}.

\item[\code{der}] number or vector which determines any inference process. By default \code{der} is \code{NULL}. If this term is \code{0}, the plot show the initial estimate. If it is \code{1} or \code{2}, it is designed for the first or second derivative, respectively.

\item[\code{points}] draw the original data into the plot. By default it is \code{TRUE}. 

\item[\code{xlab}] 
a title for the x axis. 

\item[\code{ylab}] 
a title for the y axis.

\item[\code{ylim}] the \code{y} limits of the plot.

\item[\code{main}] an overall title for the plot.

\item[\code{col}] 
a specification for the default plotting color.

\item[\code{CIcol}] 
a specification for the default confidence intervals plotting color.

\item[\code{ablinecol}] the color to be used for \code{abline}.

\item[\code{abline}] draw an horizontal line into the plot of the second derivative of the model.

\item[\code{type}] 
what type of plot should be drawn. Possible types are, \code{p} for points, \code{l} for lines, \code{o} for overplotted, etc. See details in \code{\LinkA{par}{par}}.

\item[\code{CItype}] 
what type of plot should be drawn for confidence intervals. Possible types are, \code{p} for points, \code{l} for lines, \code{o} for overplotted.

\item[\code{lwd}] the line width, a positive number, defaulting to 1.  See details in \code{\LinkA{par}{par}}.

\item[\code{CIlwd}] the line width for confidence intervals, a positive number, defaulting to 1. 

\item[\code{lty}] the line type. Line types can either be specified as an integer (0=blank, 1=solid (default), 2=dashed, 3=dotted, 4=dotdash, 5=longdash, 6=twodash).  See details in \code{\LinkA{par}{par}}.

\item[\code{CIlty}] the line type for confidence intervals. Line types can either be specified as an integer (0=blank, 1=solid (default), 2=dashed, 3=dotted, 4=dotdash, 5=longdash, 6=twodash). 

\item[\code{...}] 
other options.

\end{ldescription}
\end{Arguments}
%
\begin{Value}
simply produce a plot.
\end{Value}
%
\begin{Author}\relax
Marta Sestelo, Nora M. Villanueva and Javier Roca-Pardiñas.
\end{Author}
%
\begin{Examples}
\begin{ExampleCode}

library(NPRegfast)
data(barnacle)

################################################
# Nonparametric regression without interactions
################################################
fit<-frfast(DW~RC,data=barnacle)
plot(fit,der=0)
plot(fit,der=1,col="red",CIcol="blue",points=FALSE)

##############################################
# Nonparametric regression with interactions
##############################################
fit2<-frfast(DW~RC:F,data=barnacle)
plot(fit2)
plot(fit2,der=0,fac=2)
plot(fit2,der=1,col="red",CIcol="green")
plot(fit2,der=c(0,1),fac=c(1,2))
\end{ExampleCode}
\end{Examples}


\inputencoding{utf8}
\HeaderA{predict.frfast}{Prediction from fitted \code{frfast} model}{predict.frfast}
%
\begin{Description}\relax
Takes a fitted \code{frfast} object  and produces predictions (and optionally estimates standard errors of those predictions) from a fitted model with interactions or without interactions.
\end{Description}
%
\begin{Usage}
\begin{verbatim}
predict.frfast(model,newdata,fac=NULL,der=NULL,...)
\end{verbatim}
\end{Usage}
%
\begin{Arguments}
\begin{ldescription}
\item[\code{model}] a fitted \code{frfast} object as produced by \code{frfast()}.

\item[\code{newdata}] a data frame containing the values of the model covariates at which predictions are required.  If newdata is provided, then it should contain all the variables needed for prediction: a warning is generated if not.

\item[\code{fac}] number or vector which determines the level to take into account in the plot. By default is \code{NULL}.

\item[\code{der}] number or vector which determines any inference process. By default \code{der} is \code{NULL}. If this term is \code{0}, the plot show the initial estimate. If it is \code{1} or \code{2}, it is designed for the first or second derivative, respectively.

\end{ldescription}
\end{Arguments}
%
\begin{Details}\relax
\code{print.frfast} tries to be smart about \code{summary.frfast}.
\end{Details}
%
\begin{Value}
\code{predict.frfast} computes and returns a list containing predictions of the estimates, first and second derivative, with their 95 \% confidence intervals. 
\end{Value}
%
\begin{Author}\relax
Marta Sestelo, Nora M. Villanueva and Javier Roca-Pardiñas.
\end{Author}
%
\begin{Examples}
\begin{ExampleCode}
library(NPRegfast)
data(barnacle)

################################################
# Nonparametric regression without interactions
################################################
fit<-frfast(DW~RC,data=barnacle)

nd=data.frame(RC=c(10,14,18))
predict.frfast(fit,newdata=nd)


##############################################
# Nonparametric regression with interactions
##############################################
fit2<-frfast(DW~RC:F,data=barnacle)

nd2=data.frame(RC=c(10,15,20))
pred<-predict.frfast(fit2,newdata=nd2)
pred$Level_1$Estimation
predict.frfast(fit2,newdata=nd2,der=0,fac=2)
\end{ExampleCode}
\end{Examples}


\inputencoding{utf8}
\HeaderA{summary.frfast}{Summarizing fits of frfast class}{summary.frfast}
\aliasA{print.frfast}{summary.frfast}{print.frfast}
%
\begin{Description}\relax
Takes a fitted \code{frfast} object produced by \code{frfast()} and produces various useful
summaries from it.
\end{Description}
%
\begin{Usage}
\begin{verbatim}
summary.frfast(model)
\end{verbatim}
\end{Usage}
%
\begin{Arguments}
\begin{ldescription}
\item[\code{model}] a fitted \code{frfast} object as produced by \code{frfast()}.
\end{ldescription}
\end{Arguments}
%
\begin{Details}\relax
\code{print.frfast} tries to be smart about \code{summary.frfast}.
\end{Details}
%
\begin{Value}
\code{summary.frfast} computes and returns a list of summary information for a fitted \code{frfast} object.

\begin{ldescription}
\item[\code{model}] type of estimate.
\item[\code{h}] the kernel bandwidth smoothing parameter.
\item[\code{dp}] degree of a polynomial.
\item[\code{nboot}] number of bootstrap repeats.
\item[\code{kbin}] number of binning nodes over which the function is to be estimated.
\item[\code{n}] total number of data.
\item[\code{fmod}] factor's levels.

\end{ldescription}
\end{Value}
%
\begin{Author}\relax
Marta Sestelo, Nora M. Villanueva and Javier Roca-Pardiñas.
\end{Author}
%
\begin{Examples}
\begin{ExampleCode}
library(NPRegfast)
data(barnacle)


################################################
# Nonparametric regression without interactions
################################################
fit<-frfast(DW~RC,data=barnacle)
summary(fit)

##############################################
# Nonparametric regression with interactions
##############################################
fit2<-frfast(DW~RC:F,data=barnacle)
summary(fit2)
\end{ExampleCode}
\end{Examples}
